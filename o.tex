% !TEX program = xelatex
% % % % % % % % % % % % % % % % %
% `ctrl+F` goto 从这里开始
% % % % % % % % % % % % % % % % %
\documentclass{ecustbachelorthesis}
\renewcommand{\thesistype}{(开题报告)}
\renewcommand{\thesistitle}{华东理工大学本科生毕业论文模板}
\updatecmd
\usepackage{datetime}
\hypersetup{
  pdfinfo={
    Author={dagnaf},
    Title={\thesistitle{}\thesistype},
    CreationDate={D:20150329160000},
    ModDate={D:\pdfdate},
    Keywords={华东理工大学;本科生毕业论文;模板},
    Subject={华东理工大学本科生毕业论文模板}
  }
}
\begin{document}

\label{title:t1}
\pdfbookmark[0]{标题}{title:t1}
\mktitle{\thesistitle}{班级(学号)姓名}

\mkabstract{这篇文章本身不是论文,而是示例,仅符合论文格式。一方面这是直接使用模板生成,供使用者参考。另一方面也用于模板的基本介绍。模板包括开题报告、文献翻译和论文,项目中附带了相应文件,可以在上面直接修改。本文首先介绍华东理工大学,然后再介绍使用方法。
}{
华东理工大学,本科生毕业论文,模板
}

% #从这里开始# 写开题报告
\chapter{研究背景}
\section{标题}
\subsection{标题}
\begin{subtext}

一级标题(黑体小二号,居中,段前距为12磅),二级标题(宋体四号,段前距为12磅,段后距为0磅,接内容段前距为12磅),三级标题(黑体小四号,段前距为12磅,段后距为0磅),内容(首行缩进二格,宋体小四号)
\end{subtext}
\section{标题}
\subsection{标题}
\begin{subtext}
表(宋体小五号粗体,按章编号,例如表2.7为第2章第7个表)
\begin{center}
\setstretch{1.35}
\vspace{-20pt}
\begin{longtable}{|c|c|c|c|}
\caption{林可霉素……}
\label{tab:linc} \\

% \hline \multicolumn{1}{|l|}{} & \multicolumn{1}{l|}{徐汇} & \multicolumn{1}{l|}{奉贤} & \multicolumn{1}{l|}{金山}\\ \hline
\endfirsthead

% \multicolumn{4}{r}%
% {接表 \ref{tab:linc}} \\
% \hline \multicolumn{1}{|l|}{} & \multicolumn{1}{l|}{徐汇} & \multicolumn{1}{l|}{奉贤} & \multicolumn{1}{l|}{金山}\\ \hline
\endhead

% \hline \multicolumn{4}{r}{待续}
\endfoot

% \hline
\endlastfoot
\hline
    第1行第1列 & 第1行第2列 & 第1行第3列 & 第1行第4列\\ \hline
    第2行第1列 & 第2行第2列 & 第2行第3列 & 第2行第4列\\ \hline
    第3行第1列 & 第3行第2列 & 第3行第3列 & 第3行第4列\\ \hline

\end{longtable}
\end{center}
\vspace{-42pt}

如表\ref{tab:linc}所示,这张表格为测试表格,标注格式规范文档。注意格式要求表名位于表格正上方,所以写论文时,要把\verb$\caption{...}$放在\verb$\begin{tabular}...$上面。
\end{subtext}
\chapter{文献综述}
\section{标题}
\subsection{标题}
\begin{subtext}
数学公式(用斜体,按章编号),如公式\ref{eqn:abc}所示,这里可以用\verb$\ref{label}$来具体指代所说的公式,其中\verb$label$需要替换为上下文中设定label名字,例如在\verb$equation$中用\verb$\label{eqn:abc}$来设定该公式的标签(即\verb$label$)。
\setlength{\abovedisplayskip}{0cm}
\setlength{\belowdisplayskip}{0cm}
% \vspace{-17pt}
\begin{equation}
A=0.235B+2.68C \label{eqn:abc}
\end{equation}

另外公式最好还是自己修改行距
\end{subtext}
\section{标题}
\subsection{标题}
\begin{subtext}
图(图名位于图的正下方,用宋体小五号粗体,按章编号,图3-1为第3章第1个图)。如图\ref{fig:linc}所示,这张图为测试图,标注来自格式规范文档\cite{ecustjwc}。注意格式要求图名位于图片正下方,所以写论文时,要把\verb$\caption{...}$放在\verb$\includegraphics{...}$下面。
\setlength{\belowcaptionskip}{-10pt}
\vspace{-7pt}
\begin{figure}[!htbp]
  \centering
  \frame{\includegraphics{images/img1}}
  \vspace{-5pt}\caption{林可霉素……}
  \label{fig:linc}
\end{figure}

以上格式适用于论文、开题报告和文献翻译。参考文献使用bibtex工具,把文献按格式加入\verb$.bib$文件中,建议使用LaTeXTools编译。
\end{subtext}
\chapter{技术路线}
\chapter{进度安排}

% 最后的参考文献
\bibliography{o.bib}
\end{document}
