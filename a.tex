% !TEX program = xelatex
\documentclass{ecustbachelorthesis}
% \graphicspath{{img/}}
% \renewcommand{\thetable}{\arabic{table}}
% \renewcommand{\thefigure}{\arabic{figure}}
\input{_}
\renewcommand{\thesistype}{}
\renewcommand{\thesistitle}{\realthesistitle}
\updatecmd
\usepackage{datetime}
\hypersetup{
  pdfinfo={
    Author={\realauthorname},
    Title={\thesistitle{}\thesistype},
    CreationDate={D:20150315102933},
    ModDate={D:\pdfdate},
    Keywords={\realkeywords},
    Subject={MOOC数据结构课程动态演示系统}
  }
}

% \usepackage{showframe}
\begin{document}
% \label{title:t1}
% \pdfbookmark[0]{标题}{title:t1}
% !TEX root = ../../t.tex
\mkabstract{
我们在 \url{http://www.comp.nus.edu.sg/~stevenha/visualization} 上实现了一个统一交互式算法可视化网站,其中包括各种经典或非经典的算法。相比其他许多算法可视化的网站,我们所提供的算法可视化有以下几点优势:(1)支持各种非经典算法,这些算法现在还没有在其他网站上实现过;(2)拥有交互式特性。用户(通常为学生)可以输入他们自己的数据来测试算法;(3)提供统一的用户界面;(4)使用HTML5搭建,兼容当前可携带个人电脑,包括平板和智能手机。我们在新加坡国立大学的两个算法课程班上进行用户研究,结果显示我们的可视化系统对于一些偏向于可视化学习的学生有很大的帮助。
}{
统一,交互式,算法,可视化
}

\mktableofcontents
% % !TEX root = ../../t.tex
\chapter{AV的早期应用}
\begin{sectext}
AV在CS教育中有一段长时间的历史,最早可以追溯到1981年的``排序总结''(Baecker,Sherman,1981)和BALSA系统(Brown和Sedgewick, 1985)。之后还出现了数百种AV应用供教育者开放使用,以及数百篇关于AV的论文(AlgoViz.org文献库,2011)。好的AV应用使算法形象化,用图形展示算法中不同的状态、状态过渡的动画,用自然抽象化的方式代替数据结构中的内存地址和函数调用。

BALSA(Brown,Sedgewick,1985)、Tango(Stasko,1990)、XTango(Stasko,1992),Samba(Stasko,1997)和Polka(Stasko,2001)等早期的系统由于实现技术难以在使用者中进行传播,这也是早期教育软件常有的问题。麻省理工学院的Athena项目(Champine,1991)不仅提供交互式课程软件,同时也通过X Windows桌面系统解决了传播问题。因此20世纪80年代末到90年代初的许多AV系统都建立在X Windows基础上。但是许多教育机构的实验室内由于工作站没有足够大的功率来运行X Windows,仍然难以使用这些课程软件。

Hundhausen、Douglas和Stasko首次对AV系统教学效果进行大规模的评估(2002),包括24个小规模的元分析实验,结果发现其中的11个实验中,使用AV技术的学生数据和使用另一种AV技术或完全不使用AV技术的学生数据有显著的差异(p. 265),另外10个实验则没有明显结果,还有两个有明显结果,但是使用AV技术的学生数据不准确,最后一个同样也有显著结果,但结果反而得到的是AV技术阻碍了学生学习。由于缺少统一的知识类型标准,实验难以衡量教学成果。一些关于AV技术的研究考察学生概念性或者陈述性知识(对算法概念的理解)以及过程性知识(对算法步骤和数据操作的记忆)。而在这些研究中,只有在考察过程性知识时能够得到最显著的结果。

Hundhausen等的论文中评论过AV系统(2002),这些系统根据其教学效果都存在不同的结果。XTango的实验中,仅仅使用教科书与同时使用教科书和可视化工具这两者之间的数据在考试结果上没有差异。另一个关于用XTango和Polka(Byrne,Catrambone,Stasko,1999)作为授课工具的实验中得到的结果是:相比仅仅使用教科书的学生,看过演示动画的学生能够更好地判断出算法的下一步。而在编程方面,使用BALSA II (brown,1988)实际操作并观看过算法演示动画的学生比没有的学生表现得更好。该实验的作者指出当前研究的AV系统都是Java版本前的变形,而且学生只能在计算机实验室中使用电脑。

也许Hundhausen等(2002)的元分析中最重要的成果在于(a)学生如何使用AV对教学效果有影响,而不是学生看到了什么样的AV;以及(b)AV技术能够吸引学生主动交互时最有效果。这种关键在于交互的观点很大程度上影响了AV之后的发展。
\end{sectext}

% % !TEX root = ../../t.tex
\chapter{互联网时代下的AV}
\begin{sectext}
随着万维网和我们熟知的互联网的出现,20世纪90年代中期发生了许多改变。越来越多的教师和学生都在使用AV系统,大多是因为Java的广泛使用,还有一小部分是由于Javascript和Flash。也就大约在这个时期,越来越多的学生在校园里拥有自己的计算机。而从20世纪90年代末到21世纪初,一种新的基于JAVA的AV开发系统逐渐形成,而以前基于X Windows的AV系统在教育领域内几乎完全失去了影响力。

基于JAVA的AV系统包括JSamba(Stasko 1998)、JAWAA(Pierson,Rodger,1998)、JHAVE`(Naps,Eagan,Norton,2000)、ANIMAL(Rossling,Schuler,Freisleben,2000)和TRAKLA2(Korhonen等,2003)。这些系统的共同特点在于,作为开发AV的完整工具,能够为AV开发者提供一个编程框架。Java版本前的系统难以深入开发,虽说投入大量精力后还是能够进行开发的。相反,一些基于Java的开发系统(例如JAWAA)则注重可视化开发的便捷性,但不太重视交互性。

与Java广泛使用起到同样关键性的变化还在于出现了许多不依赖于开发系统的AV。这是因为Java不受制于操作系统或者其他类似的限制,同时互联网也利于发布软件。这就改变了开发者的自身角色。虽然JHAVE和ANIMAL按照传统开发系统,系统还是非常庞大。而其他许多项目则将AV打包为独立软件,不能用于其他AV开发中。大型非打包的系统包括数据结构浏览器(Data Structure Navigator,DSN)(Dittrich,an den Bercken,Schafer,Klein,2001),交互式数据结构可视化(Interactive Data Structure Visualization,IDSV)(Jarc,1999),算法动画(Stern,2001),可视化的数据结构(Galles,2006)和TRAKLA2(Korhonen等,2003)。

Java出现的另一个影响在于教师学生开始进行一次性的AV开发。编写AV对于学生而言是学好JAVA的一种途径,这在90年代末非常吸引人。遗憾的是大多数这些学生编写的AV难以发挥教学作用(Shaffer等,2010)。不过还是有一部分AV是通过几年持续努力而编写出的Java小应用程序,并且在某些课程上展示而受到广泛好评。例如二叉树之类(Binary Treesome)(Gustafson,Kjensli,Vold,2011),JFLAP(Rodger,2008),马里兰大学的空间索引示例(Spatial Index Demos)(Brabec,Samet,2003)和弗吉尼亚理工大学算法可视化研究小组(Virginia Tech Algorithm Visualization Research Group)(2011)。
\end{sectext}

% % !TEX root = ../o.tex
% \vspace*{-13pt}
\chapter{技术路线}
\begin{chatext}
研究将设计并实现一个基于MOOC的数据结构动态演示系统,采用web技术使得该系统可以嵌入到MOOC课程网站。系统使用的主要技术为Javascript技术,研究的主要内容为数据结构和动态演示。
\end{chatext}
\section{Javascript}
\begin{sectext}
Javascript是一种属于网络的脚本语言,已经被广泛用于Web应用开发,常用来为网页添加各式各样交互行为,具有良好的跨平台性,在大多数浏览器的支持下,能够完成许多出色的功能。

在基于MOOC的数据结构动态演示系统中,Javascript主要用来创建数据结构的动态演示,提高网页的交互性。其主要的开发平台有浏览器控制台和node.js平台。浏览器控制台可用于设计、实现过程中的调试,将最终系统的展示在浏览器页面中;而node.js开发平台是一个基于Chrome Javascript运行时平台,有自己的包管理工具,用于方便、快捷的搭建易于扩展的网页应用。
\end{sectext}
\section{数据结构和动态演示}
\begin{sectext}
系统中涉及的数据结构来源于MOOC数据结构课程的教学大纲,同时也参考传统教学的教材中所涉及到的基本数据结构和算法。系统中演示数据结构的参考源代码采用相应教学语言(C语言)进行编写,这些代码与系统中的动画同步演示,可以帮助学习者加深对代码的理解,在实际编程中更好地应用数据结构。演示的内容主要由数据结构和算法构成,其中数据结构不仅仅是数据结构本身的概念,而是通过演示其在算法或实际应用中的初始化、具体操作步骤等以达到课程更加生动有趣的效果。

系统中演示的数据结构及其应用如下:
\begin{itemlist}
\item 栈:算术表达式。
\item 队列:杨辉三角。
\item 矩阵:稀疏矩阵的存储压缩。
\item 树:哈夫曼编码、平衡二叉树。
\item 图:拓扑排序、强连通分量、Dijkstra最短路、Prim最小生成树。
\item 散列表:开地址、链地址解决冲突。
\end{itemlist}

系统中演示的算法如下:
\begin{itemlist}
\item 排序:归并排序、快速排序。
\item 查找:顺序查找、二分查找。
\end{itemlist}

动态演示的实现包括实现其界面绘制、图形绘制和事件的处理,这些则全部由Javascript技术支持。

首先,演示界面相对统一,对多个不同内核的浏览器进行兼容。既可以通过编写简单的HTML和CSS实现,也可以加入相应的Javascript UI库对网页调整。

其次,数据结构图形对于不同的数据结构会有所差异,但一般都是由基本图形构成,即线段、弧线、矩形、圆形等,这些在HTML页面上的绘制可以通过D3.js库完成。D3.js是最流行的可视化库之一,基于数据对HTML文档树进行操作,常用于将数据转化为交互性的图形,支持SVG和Canvas两种模式。

最后,事件的处理,也就是页面交互以及动画过程,需要结合Javascript的设计模式。发布者—订阅者模式,或者是观察者模式,都是处理事件常用的模式。这种模式定义了一种一对多的依赖关系,可以有效的解决对象互相依赖过程中产生的副作用,维护相关对象的一致性。

以上三个模块,由相互独立的Javascript库完成,借助node.js的模块化开发的特点,将模块整合成一个系统的模块,单元间可以共享资源,例如:演示界面的元素等。而每个系统的模块作为一种数据结构的动态演示,可以嵌入到MOOC的课程网站中,也可以独立在单一的网页中显示,从而搭建基于MOOC的数据结构课程动态演示系统。
\end{sectext}
%

% % !TEX root = ../o.tex
\chapter{进度安排}
\begin{chatext}
% !TEX root = ../o.tex
% c4t1.tex

\begin{center}
\setstretch{1.35}
\vspace{-20pt}
\vspace{-14pt}
\begin{longtable}{l r l l l l}
% \caption{算法可视化的效果}
\label{tab:3} \\
% \hline \multicolumn{1}{l}{积极方面} & \multicolumn{1}{l}{消极方面} \\ \hline
\endfirsthead
% \multicolumn{2}{r}%
% {接表 \ref{tab:3}} \\
% \hline \multicolumn{1}{l}{积极方面} & \multicolumn{1}{l}{消极方面} \\ \hline
\endhead
% \hline \multicolumn{2}{r}{待续}
\endfoot
% \hline
\endlastfoot

2014年 & 12月 & -- & 2015年 & 3月 & 查阅文献、研究课题、完成文献翻译和开题报告\\
2015年 & 3月 & -- & 2015年 & 4月 & 进行需求分析、设计系统\\
2015年 & 4月 & -- & 2015年 & 5月 & 实现系统、完成测试\\
2015年 & 5月 & -- & 2015年 & 6月 & 完成毕业论文、参加答辩\\
\end{longtable}
\end{center}

\end{chatext}
\vspace{-44pt}

\chapter{引言}
\section{背景}
\subsection{国内}
\subsection{国外}
% !TEX root = ../a.tex
\acknowledgment

\nocite{*}
\bibliography{a.bib}
\end{document}


% 毕业设计期间必须做出工程产品一部分或相对完整的工程原型系统。现代软件工程非常强调撰写文档,在开发期间,各阶段都要撰写文档。毕业论文属技术报告型,报告本课题所依据的原理、规范和实现模型所需的环境支持,总体实现方案,各(子)部分的设计与实现,与外部的接口等。但不等于文档集合,毕业论文应该带有论证性。如这个课题当今有几种实现方案、为什么要选择此种方案、本方案有何优缺点等。
% 论文要突出重点,对核心、独创部分尽可能地详细。要强调系统性,即从分析、设计、实现到测试,每个阶段的重点技术要清楚。特别是测试结果和使用情况是工程型课题最为增色之处,切不可略去。
% 对工程型论文,强调应用性。例如完成一个不太大的实际项目或在某一个较大的项目中设计并完成一个模块(如应用软件、工具软件或自行设计的板卡、接口等等),然后以工程项目总结或科研报告、或已发表的论文的综合扩展等形式完成论文。
% 论文应重点收集整理课题的立题背景,需求分析,平台选型,使用的开发工具,系统体系结构,程序模块调用关系,数据结构,算法,实验或测试等内容 。 论文结构一般安排如下:
% (1)引言或背景 (概述题目背景,实现情况,自己开发的内容或模块) 。
% (2)一般谈课题意义,综述已有成果,如“谁谁在文献某某中做了什么工作,谁谁在文献某某中有
% 什么突出贡献”,用“但是”一转,分析存在问题,引出自己工作必要性、意义和价值、创新点和主要思
% 想、方法和结果。然后用“本文组织如下:第一章....,第四章.....” 作为这段结束。
% (3)课题的分析设计(包括概要设计和详细设计,应强调系统的整体性,重点描述项目的整体框架,
% 功能,开发工具,突出自己工作在整体中的位置,对概要设计的细化等。这部分内容是全文的重点)。
% (4)主要实现及功能的描述(包括模块调用关系,数据结构,算法说明,依据内容多少此部分可安
% 排两到三节)
% 。
% (5)实验或测试(描述数据库设计结果,代码开发原理和过程,实现中遇到和解决的主要问题,系
% 统的测试,今后的维护和改进等)。
% (6)总结(全文总结及展望)。
% 这类工作应进行系统演示。
