% % % % % % % % % % % % % % % % %
% `ctrl+F` goto 从这里开始
% % % % % % % % % % % % % % % % %

\documentclass{ecustbachelorthesis}

\begin{document}
% 文档初始化
\docinit
\pagenumbering{Roman}
\setlength{\baselineskip}{18pt}

% 中文摘要
\begin{abstractzh}
\makeatletter\me@abstractzh\makeatother
\end{abstractzh}

% 英文摘要
\begin{abstracten}
\makeatletter\me@abstracten\makeatother
\end{abstracten}

% 目录
\mktableofcontents
\clearpage
\pagenumbering{arabic}

% #从这里开始# 写论文
\section{简单介绍}
\subsection{华东理工大学}
华东理工大学原名华东化工学院... 今天,华东理工大学正昂首阔步,在未来十年或者更长一段时间,朝着把学校建设成为国际知名、特色鲜明、多学科高水平研究型大学的总体目标前进\cite{ecust}。
\begin{table}[!htbp]
  \centering
  \caption{华东理工大学各校区}
  \label{tab:ecust}
  \begin{tabular}{|l|l|l|l|}\hline
        & 徐汇        & 奉贤       & 金山\\ \hline
    地址 & 梅陇路130号 & 海思路999号 & 学府路1000号\\ \hline
    邮编 & 200237     & 201424    & 201512\\
    \hline
  \end{tabular}
\end{table}

学校有三个校区分别是徐汇校区、奉贤校区和金山科技园区,具体地址如\ref{tab:ecust}所示,基本上本科生3年在奉贤,1年在徐汇。最后大四这一年搬到徐汇后,相比之下,更喜欢徐汇校区。树木绿化都很有味道,校园很美,而且感觉天气多数都是蓝天白云。

虽然宿舍很小,但是生活的还是很愉快的。为什么呢?既然你诚心诚意的问了,我就大发慈悲的告诉你,为了防止世界被破坏,为了守护世界的和平,贯彻爱与真实的邪恶,可爱又迷人的反派角色,武藏,小次郎,我们是穿梭在银河中的火箭队,白洞,白色的明天在等着我们,就是这样\cite{pokemon}。
\subsection{LaTeX}
在LaTeX中使用\verb${\LaTeX}$可以显示出如图\ref{fig:latex}的文字\cite{latex}。那么这一小节就这样结束了,关于LaTeX的详细信息可以在网上查阅相关资料。
\begin{figure}[!htbp]
  \centering
  \includegraphics[width=150pt]{latex}
  \caption{LaTeX文字图标}
  \label{fig:latex}
\end{figure}
\section{模板使用}
\subsection{信息填写}
整个毕业论文分为论文\verb$[thesis]$、开题报告\verb$[opening]$和文献翻译\verb$[translation]$。通过加载模板时传入参数来分别使用。同时这些文档在格式上有共同点,还有一些共享的常量,为了保证各文档之间内容的一致性,在开始撰写论文之前,需要在\verb$.cfg$文件中填写好这些信息。这些信息包括:
\begin{itemlist}
  \item 标题 \verb$\me@title$
  \item 班级、学号、姓名 \verb$\me@class \me@number \me@name$
  \item 中文摘要和关键词 \verb$\me@abstractzh \me@keywordszh$
  \item 英文摘要和关键词 \verb$\me@abstracten \me@keywordsen$
\end{itemlist}

例如,打开\verb$information.cfg$文件重新定义这些常量,然后再参照下面的几点完成三种文档的撰写:
\begin{itemlist}
  \item 在开题报告、文献翻译中用\verb$\makethesistitle\makemyinfo$这样加入标题和个人信息。
  \item 在开题报告中用\verb$\begin{abstract}[...]...\end{abstract}$这样加入摘要,其中参数可选,环境内容也可选。默认参数和环境内容是从\verb$.cfg$中读取的关键词和中文摘要。
  \item 在论文中用\verb$\begin{abstractzh}...$或者\verb$\begin{abstracten}...$这样加入中英文摘要。
\end{itemlist}
\subsection{图、表、公式和列表}
这一部分待补全,请直接参考本文模板或者相应内容的使用方法。
\subsection{添加引用}
参考文献使用BibTeX管理,使用的是gbt7714-2005参考文献格式\cite{gbt}。具体操作如下:
\begin{itemlist}
  \item 在\verb$.tex$文件中使用引用,并在\verb$\bibliography{...}$中指定\verb$.bib$文件。
  \item 根据参考文献格式,添加条目至(1)中所指定的\verb$.bib$文件中。
  \item 使用工具编译一次\verb$.tex$文件 \verb$xelatex [filename].tex$
  \item 使用工具编译生成的\verb$.aux$文件 \verb$bibtex [filename].aux$
  \item 使用工具编译两次\verb$.tex$文件 \verb$xelatex [filename].tex$
\end{itemlist}

上述过程也可以通过\verb$Makefile$文件或者\verb$.bat$文件实现单命令执行。
\section{一级标题}
\subsection{二级标题}
\subsubsection{三级标题}
内容
% 最后的参考文献和致谢
\bibliography{thesis}
% 再 #从这里开始# 写致谢
\acknowledgement
致谢另起一页,与正文连续编页码,“致谢”两个字居中,用黑体小二号,段前设置为0磅,段后设置为12磅。致谢内容应实事求是,客观公正。\cite{ecustack}
\end{document}

