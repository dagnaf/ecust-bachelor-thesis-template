% % % % % % % % % % % % % % % % %
% `ctrl+F` goto 从这里开始
% % % % % % % % % % % % % % % % %

\documentclass[translation]{ecustbachelorthesis}

\begin{document}
% 文档初始化
\docinit
% 添加标题
\makethesistitle\makemyinfo
% #从这里开始# 写文献翻译
\makethesistitle[外文文献中文标题]\makemyinfo[外文文献作者]
\begin{abstract}[外文文献的关键词]
以上的外文文献中文标题等都是可选,具体删除请参考下文。文献翻译的具体格式在规范文档中也没有明确给出,该模板按照论文正文格式排版。
\end{abstract}
\section{文献的第一章节}
要删除外文文献中文标题等,可以看到\verb$.tex$代码中有\verb$\makethesistitle[...]...$和\verb$\begin{abstract}[...]...$这两部分。只要删除或者注释即可。注意不要删除论文本身的标题和个人信息。

注意下面的参考文献,如果没有参考文献,则删除\verb$.tex$文件最后的\verb$\nocite{*}$和\verb$\bibliography{...}$,而且编译生成pdf时只要执行\verb$xelatex filename.tex$即可。
% 最后的参考文献,如没有则删除\nocite 和 \bibliography,同时注意生成pdf时只需要执行编译tex的命令
\nocite{*}
\bibliography{translation}
这里没有引用参考文献,但用于示例,还是显示了\verb$参考文献$的标题。
\end{document}
