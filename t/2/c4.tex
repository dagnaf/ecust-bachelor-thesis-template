% !TEX root = ../../t.tex
\chapter{一些可视化工具}
\begin{sectext}
在这一小节中,作者详细描述几个可视化工具或系统,这些工具和系统在推进教学上有所记录,而且由于大多数采用了观看层级以上的交互特性,在对照实验结果中显示了学生使用前后数据上的差异。
\end{sectext}
\section{TRAKLA2}
\begin{sectext}
TRAKLA2在所有芬兰大学的常规教学中得到应用,是最广泛使用的AV集合之一。TRAKLA2有许多响应层级交互式的优秀例子,可以控制算法中数据结构的表现形式,而学生可以通过拖放GUI元素``创建''数据结构。TRAKLA2的练习要求学生操作数据结构的状态从而得到某些结果,比如说要创建树形结构,学生可能不断地拖放新元素到树中的指定位置。或者学生可以通过观察单步执行(也称之为参考答案)来理解算法。为了提供交互式的反复练习,大部分练习都有文本教程和解释算法的伪代码。

Laakso等(2005)在芬兰2所大学说明了数据结构与算法课程中使用TRAKLA2练习的情况。TRAKLA2练习贯穿于课堂(封闭实验室)练习、补课以及上课环节中,同时也出现在期终测试(占20\%)和期中测试(其中TRAKLA2练习5题,占50\%)。课程中引入TRAKLA2的一年后学生的上课积极性全面提高,不再仅仅是课堂练习的部分。课堂练习的平均表现从54.5\%提高到60.3\%(完成的练习数量)。学生通过在线调查反馈对于TRAKLA2的看法,80\%认为很好。而在使用一年后,认为TRAKLA2非常适合教学的学生数量显著上升,94\%认为其中的练习加快了学习过程。另一方面,学生也不介意获取工具和练习算法的方式,但更倾向与一种混合(在线课堂与传统课堂)的体验。这一结果同Levy和Ben-Ari(2007)得出的AV系统应该与现有课堂结合起来相一致,之后会再讨论到这点。

在成功的AV中,易用性对学习目标起到关键作用。通常学生在使用AV系统时需要更多的时间。但这可能是因为由于系统缺乏易用性(Pane,Corbett,John,1996)学生需要一定时间习惯。而至于TRAKLA2,有研究通过一系列调查问卷,并在一个易用的实验室进行观察(Laakso等,2005)发现并没有严重的易用性问题。学生需要在15分钟内完成所有练习,其中80\%的时间用于解题,而14\%的时间用于熟悉系统界面。这项重要的发现表明通常习惯给定的AV需要短则几分钟长则一、两个小时的时间。

TRAKLA2也用于评估EET对于学生表现的影响。Lassko,Myller和Korhonen(2009)研究了结对学习时使用EET应变层级的学生是否比使用EET对比层级的学生表现更好。第一年学生分成两组,每组都使用相同的文本材料。控制组(EET应变层级)使用TRAKLA2练习,对比组(EET对比层级)只是观察有同样信息的AV。所有的学生都独立参加了试前测试并自由结对,之后用45分钟学习材料并结对(使用纸和铅笔)解题。同时所有的学生都需要参加试后测试,并且用视频记录其过程。从第一个对试前测试和试后测试的分析中不能得到两组之间数据的显著差异,但从第二个对视频的分析中可以得到:一些在控制组的学生并非按实验预期一样使用TRAKLA2,只是看着参考答案而没有选择去解决TRAKLA2中的练习,因此这也是属于EET对比层级。于是控制组内使用对比层级交互的同学重新作为第三组,再分析得分,最后得到在控制组中使用应变层级交互的学生比使用对比层级交互的学生表现更好。
\end{sectext}
\section{JHAVE}
\begin{sectext}
JHAVE(Naps等,2000)一种AV开发系统,旨在相对简化AV开发者制作带有内置评估功能的动态幻灯片,可以在演示过程中向用户弹出问题。JHAVE的界面包括可视化面板和伪代码面板,通常还有一段关于算法的简短文字教程。信息页面中可以导入图片,用于显示算法的流程图。JHAVE内有非常多的AV,已经得到广泛使用。

Lahtinen和Ahoniemi(2009)展开了``快速启动''实验,向没有编程经验的同学开设可视化的CS1课程,并基于以下三点:

\begin{itemlist}
\item 课程从一个真实的编程问题入手,在解决问题中解释数据结构、伪代码、流程图和编程思想。课堂中使用可视化工具作为教学材料。

\item 用伪代码和流程图表示问题的解决方案,伪代码和英语描述类似。

\item 学生与算法(测试、调试等)进行互动。
\end{itemlist}

Lahtinen和Ahoniemi(2009)认为大多说AV系统假定学生已经很熟悉系统使用的编程语言。而这个实验是针对几乎没有任何编程经验的学生,就要用到没有语法限制的AV系统,其灵活性可以显示系统中算法的流程图,所以选择了JHAVE作为可视化工具。快速启动实验要求学生在课程初设计一个算法,称为``不成熟的算法''。之后对可视化后的算法进行测试并修改,最后在JHAVE的可视化中达到``成熟的算法''。实验中,系统与学生的交互性覆盖了观察、响应、应变、构建和演示层级。对于快速启动/JHAVE的评估表明86\%的学生认为可视化有助于学习,相比有编程经验的学生,没有编程经验的学生认为更有利于学习。这次有趣的实验把AV整合到教学方法中,JHAVE贯穿学生的整个课程,由于系统的可伸缩特点,将接近英语的伪代码和图像构造的算法流程图打包整合到JAVA小应用程序中,这也提供了一种使用AV工具的新途径。

JHAVE-POP(Furcy,2007)是JHAVE的一个插件,用于练习基本指针和链表操作,可以根据用户输入的C++或JAVA的代码片段,在程序声明执行时一步一步可视化地生成内存中的内容。学生所反馈的调查问卷(Furcy,2009)显示``JHAVEE-POP具体化指针的形象,加深了学生的理解'',同时JHAVE-EPOP``能够帮助学生理解并调试程序。''在这一小节中,作者详细描述几个可视化工具或系统,这些工具和系统在推进教学上有所记录,而且由于大多数采用了观看层级以上的交互特性,在对照实验结果中显示了学生使用前后数据上的差异。
\end{sectext}
\section{ALVIS}
\begin{sectext}
ALVIS(Hundhausen,Brown,2005)是一个程序开发环境,使用SALASA类伪代码语言编写程序,支持分镜功能。Hundhausen和Brown(2008)为了让有过一学期编程经验的学生在交互过程中覆盖五个层级而开展了教学实验。在实验中,学生结对使用ALVIS开发课堂中学到的算法的可视化,与同学和老师讨论并展示结果。Hundhausen和Brown把学生分为两组来评估系统,其中一组使用ALVIS另一组使用文本工具(钢笔、纸头、胶带等)。所有学生在文本编辑器(文本工具小组)或者ALVIS上使用SALSA语言编写,由摄像头进行录像。在实验结束后,对其所用到的工具进行回收,同时所有学生都需要描述自己的感受,使用文本工具的学生还需要参加采访。两组学生都用大部分时间编码,相比使用ALVIS的学生,使用文本工具的学生与助教讨论的时间更多,但所编写的每个算法代码中的错误却多出将近一倍。Hundhausen和Brown认为无论用哪种工具学生都可以学到互相讨论和批判思考。整个实验发现ALVIS能够帮助学生快速编写出语义错误较少的代码,同时可以调动学生参与到课堂,积极讨论算法。
\end{sectext}
\section{弗吉尼亚理工大学哈希教程}
\begin{sectext}
弗吉尼亚理工大学哈希教程(弗吉尼亚理工大学算法可视化研究小组网站,2011)并不只是一个简单的AV,教程涉及CS中最重要的主题,哈希搜索的概念,内容多达一整本教科书,其中包含的AV(以Java小应用程序的形式)嵌入在HTML文本中进行显示。除了展示基本的哈希处理过程,学生可以用附加的小应用程序观察教程中不同算法相对的性能差异。

2008年和2009年在大二年级开设了不同的数据结构与算法课程,学生分别都学习了哈希算法。第一种课程中,使用一周的标准课本与授课,与前几周一样,而在另一种课程中用一周的时间用同样的课本学习带有AV的网上教程,课本在两个对比课程中是完全一样的,但网上教程在AV中补充了更多的文字。

在两次实验中,两门课程都在一周结束后进行测验,测验结果表明使用网上教程的课程相比标准授课数据上有明显的优势。也就是说网上教程不仅仅和授课一样有效(对于远程学习意义深远),而且在恰当的交互体验下,计算机教学比授课教学(被动形式)效果更好。但是研究还需要考虑在课堂听讲和在实验室学习教程这样的对比环境对结果有多大的影响。如果一个学生只是自己阅读材料,学习AV,由于不同的自律表现,不能保证有充足的时间和精力,这样试验结果可能就完全不一样了。同样在对比环境中,相比只是自己阅读课本,阅读课本后再听课也可能会产生不同的结果。
\end{sectext}
\section{AlViE}
\begin{sectext}
和JHAVE一样,AlViE(AlViE,2011)是一个测试后的算法可视化工具,可以在一边执行算法一边显示可视化内容。也就是说可视化由脚本驱动,脚本由不同的方式生成,包括工具程序执行的一个复杂算法或模拟。AlViE由Java编写,用XML描述执行算法中相关事件和数据结构。

Crescenzi和Nocentini(2007)在两年的数据结构和算法课程教学中反复使用AlViE,让学生与系统的交互层级达到最大。第一年的实验的交互覆盖了无观察、观察、构建和演示层级,课程系统地使用AlViE进行算法和数据结构的教学,回家作业要求在没有可视化工具下实现算法,期末项目要求使用AlViE编写特定算法的动画并想同学和老师展示。根据学生反馈,70\%认为期末测试中使用AlViE有用,30\%则认为非常有用。同时有90\%认为动画演示有助于理解算法流程,所有学生都觉得可视化具有教学价值。

第二年改进了之前的实验,基于实验出版了一本纸制教科书(Crescenzi、Gambosi,Grossi,2006),并加入了交互中的应变层级。这本书是AlViE的延伸,描述了算法和数据结构,其中的插图均来自AlViE的图形界面。阅读时可以看到所有算法的执行过程。现在意大利的几所大学已经采用了这本书和AlVie。
\end{sectext}
\section{Alice}
\begin{sectext}
Alice(Alice,2011)是一个3D交互式编程环境,通过简洁的图形化界面让初学者接触面向对象编程。用户可以创建虚拟世界并拖拽物体到主窗口中,也可以编写脚本控制物体。Alice操作简单,以讲故事的交互形式,赢得了许多CS学校的青睐,已经应用于不同层次的编程导论课程中,可面向CS专业、非CS专业、初中和高中学生。

Moskal、Luri和Cooper(2004)研究了使用Alice是否能够CS1的学生表现,具体的问题包括:Alice能否有效提高CS专业学生的升级率?Alice能否让学生相信自己的才能可以在CS领域取得成功?实验中所有学生都是CS专业,分为三组:(a)控制组,几乎没有编程经验并参加基于Alice课程的学生,同时;(b)对比组1,几乎没有编程经验但没有参加基于Alice课程的学生;(c)对比组2,有编程经验且没有参加基于Alice课程的学生。实验课程中,控制组的学生取得了3.0±0.8的GPA,对比组1的学生取得了1.9±1.3的GPA,控制组2取得了3.0±1.2的GPA。这样的结果有可能十分重要,因为过去相比有编程经验的新生,几乎没有编程经验的新生在CS课程中表现更差,升级率更低。过去两年的实验中,控制组的升级率达到88\%,控制组1达到47\%,控制组2达到75\%。对学生态度的评估显示没有编程经验也没有参加基于Alice课程的学生``在CS1课程后更加抵触计算机科学的创造性''(p. 78)。整个实验表明Alice能够有效提高学生表现和升级率,同时改善学生对CS的看法。
\end{sectext}
\section{Jeliot}
\begin{sectext}
Jeliot(Levy等,2003)用于高中Java编程的教学,是一种PV工具,可以直接观看抽象算法对应实际程序的操作而不是内嵌的展示图,还可以在界面中选择单步执行程序。学生使用时一边查阅程序源代码,一边观看自动生成的动画。PV系统用于生成可视化内容,展现实际程序的具体操作,而无需程序员写程序那样多的精力。但系统的缺点在于不能简单分离无关细节并聚焦算法的高阶部分或指定的关键部分。

Moreno、Sutinen、Bednarik和Myller(2007)用Jeliot中的可视化显示矛盾动画,认为这样可以暴露学生对编程概念的误解。矛盾动画扩展了交互方式,响应层级中要求学生指出当前动画中的错误,应变层级中要求学生纠正错误,构建层级和演示层级则要求学生构建一个矛盾动画并向同学展示。Moreno、Joy、Myller和Sutinen(2010)开发了Jeliot ConAn系统,系统基于Jeliot的AV系统,用于生成包含错误的动画。

为了强化学生的构思模型,Ma、Ferguson、Roper、Ross和Wood(2009)提出四阶段模型,用于检测并修正编程概念上错误的构思模型。

\begin{itemlist}
\item 准备工作:找到已有的错误构思模型

\item 感知错误:引发已有构思模型的冲突后修改模型

\item 构建模型:借助可视化构建正确的构思模型

\item 实际应用:(通常通过解决编程问题)测试新构建的构思模型
\end{itemlist}
为了测试构思模型,每个学生都登录系统,系统列出必须掌握的所有编程概念的大纲,每一条概念都有一系列习题,每一道习题都包含一个认知问题、有相关Jeliot系统中的可视化材料和最后一个测试概念理解的问题。如果学生在人质问题上回答错误,就需要运行附带的可视化动画,识别出自己的模型与程序执行之间的差异。在这一阶段,学生可以请教老师帮助深入理解算法,之后再运行相似的程序测试不同数据的同一个概念。实验从三个方面对模型进行评估:条件和循环、作用域和参数传递以及对象引用的赋值。对与所有44个学生中能够正确理解概念的人数,在条件和循环方面,实验前有23\%,实验后有73\%;在作用域和参数传递方面,实验前只有19\%,实验后则有85\%。Ma等(2009)发现四阶段模型在简单的概念上是正确的,但对于复杂概念难以得到准确结果(例如对象引用的赋值概念)。
\end{sectext}
\section{ViLLE}
\begin{sectext}
ViLLE(Rajala、Laasko、Kaila,Salakowski,2007)是芬兰的Turku大学开发的程序可视化工具,用于辅助教学或者可视化教师或学生创建的示例程序。系统支持多种多种编程语言(包括Java、C++和可扩展的伪代码),内置编辑器响应交互式的弹窗测试问答。学生可以跟踪程序状态和数据结构的变化。

为评估ViLLE的有效性,学生在第一年编程课程中使用系统(Rajala、Laasko、Kaila,Salakowski,2007)。实验将学生分为两组进行两小时的计算机上机。两组学生都参与了试前测试,回答程序输出、状态等问题后,一组学生在ViLLE下学习编程教程,而另一组则用教科书学习,最后同时参加了试后测试,题目同试前测试一样,额外还有两个关于使用给定语句完成程序的问题。实验在知识获取方面的结果并不明显,但是在使用ViLLE的小组中,之前没有编程经验的学生比有编程经验的学生获取的知识更多。虽然缩小了编程新手和老练程序员之间的差距,但最后测试总分上仍有明显的差异。另外在高中的ViLLE实验中(Kaila、Rajala、Laakso,Salakoski,2009),学生可以从Moddle课程管理系统获取所有的课堂材料。期中测试中,使用ViLLE的学生相比对比组表现更好(数据上很明显),认为ViLLE对于跟踪程序执行和学习编程技巧上非常有效。
\end{sectext}
\section{jGRASP}
\begin{sectext}
jGRASP(Hendrix等,2004)是一个完整的程序开发环境,拥有同步``对象浏览器''功能,将对象和数据结构状态可视化,现在已经应用于使用Java的CS2课程实验中。Jain、Cross、Hendrix和Barowski(2006)在Auburn大学进行一些列研究,分析了jGRASP的教学优势,即使用jGRASP的数据结构浏览器是否能让学生写出错误少的代码、准确检测并修改不是语法性质的错误。其中的两次实验在实验室内进行,分别把学生分为两组,一组只使用JGRASP调试器而另一组调试器和对象浏览器。实验要求学生实现单向链表的四个基本操作,结果显示在准确度上两组数据有明显差异,控制组的平均准确度为6.34,而对比组为4.48。研究认为在大部分(95\%)情况下,jGRASP能够提高准确度并减少编程实现数据结构的时间(pp. 35)。另外在第二次实验还要求学生在实现单向链表函数的同时,检测并修改Java程序中的错误(一共九个),控制组平均检测出的错误为6.8,成功改正的错误为5.6,新引入的错误仅为0.65,而对比组则为平均检测错出的误数为5.6,成功改正的错误为4.2,新引入的错误为1.3。但控制组平均使用时间更多一点,完成实验耗时88.23分钟而对比组为87.6分钟。
\end{sectext}
\section{JFLAP}
\begin{sectext}
JFLAP(Rodger,2008)可以辅助教学形式语言和自动机理论,尽管这些都是高年级课程,但系统已经用于新生第一学期的形式语言课程中。学生可以使用JFLAP创建并模拟可视化后的自动机,还可以分析语法中的字符串,将非确定有限自动机转换为确定性有限自动机。

在两年的实验中,第一年有12所大学,第二年有14所大学应用并研究JFLAP的效果(Rodger等,2009)。实验主要研究以下问题:JFLAP加快学习过程的效果有多大?JFLAP对于形式语言和自动机课程有哪些额外的作用?在第一年里,大部分教师将其用于回家作业和课堂展示;没有教师用于测验。结果显示学生中55\%在准备测验时使用系统,94\%有充足的时间学习如何使用系统,64\%表示系统有助于提高成绩,83\%表示系统相比书写更加方便,但50\%认为即便没有JSFLAP也能达到一样的效果。而在第二年里,试前测试和试后测试都有所减少,调查显示学生中,32\%在准备测验时使用系统的时间超过复习总时间的五分之一,29\%使用系统研究扩展问题,63\%表示课程更加有趣,72\%表示会主动参与课堂互动。
\end{sectext}
