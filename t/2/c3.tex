% !TEX root = ../../t.tex
\chapter{交互带来的影响}
\begin{sectext}
从21世纪头10年的早期到中期出现了一系列影响深远的工作组报告,这些报告是例年ACM计算机科学教育大会创新技术的一部分(Naps等,2003b,Naps等,2003b,Rossling等,2006)。Naps等在报告中对Hundhausen等(2002)在AV中与学生积极交互的必要性方面加以补充,定义了一种AV与学生交互层级类型的分类,包括以下6种分类:

\begin{itemlist}
\item 无观看层级即不使用AV

\item 观看层级能显示AV并控制执行速度

\item 响应层级能回答与当前可视化内容有关的问题

\item 应变层级能执行输入的数据

\item 构建层级能可视化自己编写的算法

\item 演示层级能收集学生的反馈
\end{itemlist}

这种学生交互性分类的层次很接近Bloom层次(Bloom,Krathwohl,1956)。

Urquiza-Fuentes和Velazquez-Iturbide(2009)展开了一次新的元分析来评估AV系统在教学上的重要性,评估围绕已知的教育利益,以Naps等(2003b)定义的层级为衡量,发现与无观看层级相比观看层级并不能促进学习理解。这与之前Hundhausen等的结论(2002)是一致的。而且遗憾的是,Shaffer等(2010)在报告中提到大部分AV系统仍然处于观看层级。

Urquiza-Fuentes 和 Velazquez-Iturbide发现相比使用观看层级的AV系统,学生在响应层级上的交互能够促进知识学习。而应变层级比响应层级有更好的效果。构建层级能够编写AV,相比应变层级更能提高学习成果(虽然需要更多的时间和精力),但在知识学习上没有演示层级效果好。

Urquiza-Fuentes和Velazquez-Iturbide从成功的大型AV系统中引用了具体示例。JHAVE(Naps等,2000)AV系统由一个图形化界面、算法的文字信息和一些学习问题组成。JSamba(Stasko,1998)允许教师或学生编写脚本自定义数据结构。TRAKLA2(Korhonen等,2003)中包含了练习题,练习中学生通过自己执行单步执行演示熟悉算法,而系统也通过向学生提问算法的下一步来评估。Alice(Cooper,Dann,Pausch,2000)实现了3D虚拟空间中对象的可视化。jGRASP(Hendrix,Cross,Barowski,2004)允许教师开发具体的可视化以达到预期的目标(即对象的显示更贴近学生的喜好)

Urquiza-Fuentes和Velazquez-Iturbide还提到AV系统成功的共同要素,这些系统都有叙述内容和文本解释,可视化的步骤加以解释后可以在交互的观看层级上提高学习效果。而响应层级系统的特点在于获取学生行为的反馈,让学生在交互中表现出对于知识的熟练程度。另外要达到常时间使用系统,学生可以编写或自定义一个AV(构建层级的交互)。优秀的AV系统同样还支持一些高级特性,比如说用功能丰富的界面控制可视化。

Myller,Bednarick,Sutina和Ben-Ari(2009)研究了协作学习中交互的影响,对Naps等的交互分类进行补充,更清晰地捕捉学生的不同行为。研究补充了4种层级,其中包括介于观看层级和响应层级之间的对比层级和输入层级。这种扩展交互分类(EET)同样可以在使用可视化工具时指导学生互相合作,Myller等(2009)针对这种看法进行实验,在实验环节中让学生结对学习,并在不同交互层级上使用程序可视化工具BlueJ(Sanders,Heeler,Spradling,2001)和Jeliot(Levy,Ben-Ari,Uronen,2003),同时观察记录所有学生的交流,发现EET层级与学生交互量正相关。可见对于交互性是成功的人机合作与协作共进的重要因素之一。
\end{sectext}
