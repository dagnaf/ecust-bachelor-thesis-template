% !TEX root = ../../t.tex
\mkabstract{
计算机科学教学核心在于充分理解动态处理过程,例如实现算法的过程或者计算实体间的数据流动过程。这些动态处理过程不适合用文本、图像等静态媒介来解释,也难以在授课中传达。本文的作者基于现有记录,围绕推动教学的工具,研究了计算机科学教育中可视化的历史。其后还讨论了计算技术影响可视化工具发展的变化以及近期技术的转变推动在线超文本教科书的发展变化。
}{
算法可视化,数据结构可视化,程序可视化,电子教科书,超文本教科书
}
