% !TEX root = ../../t.tex
\chapter{AV的早期应用}
\begin{sectext}
AV在CS教育中有一段长时间的历史,最早可以追溯到1981年的``排序总结''(Baecker,Sherman,1981)和BALSA系统(Brown和Sedgewick, 1985)。之后还出现了数百种AV应用供教育者开放使用,以及数百篇关于AV的论文(AlgoViz.org文献库,2011)。好的AV应用使算法形象化,用图形展示算法中不同的状态、状态过渡的动画,用自然抽象化的方式代替数据结构中的内存地址和函数调用。

BALSA(Brown,Sedgewick,1985)、Tango(Stasko,1990)、XTango(Stasko,1992),Samba(Stasko,1997)和Polka(Stasko,2001)等早期的系统由于实现技术难以在使用者中进行传播,这也是早期教育软件常有的问题。麻省理工学院的Athena项目(Champine,1991)不仅提供交互式课程软件,同时也通过X Windows桌面系统解决了传播问题。因此20世纪80年代末到90年代初的许多AV系统都建立在X Windows基础上。但是许多教育机构的实验室内由于工作站没有足够大的功率来运行X Windows,仍然难以使用这些课程软件。

Hundhausen、Douglas和Stasko首次对AV系统教学效果进行大规模的评估(2002),包括24个小规模的元分析实验,结果发现其中的11个实验中,使用AV技术的学生数据和使用另一种AV技术或完全不使用AV技术的学生数据有显著的差异(p. 265),另外10个实验则没有明显结果,还有两个有明显结果,但是使用AV技术的学生数据不准确,最后一个同样也有显著结果,但结果反而得到的是AV技术阻碍了学生学习。由于缺少统一的知识类型标准,实验难以衡量教学成果。一些关于AV技术的研究考察学生概念性或者陈述性知识(对算法概念的理解)以及过程性知识(对算法步骤和数据操作的记忆)。而在这些研究中,只有在考察过程性知识时能够得到最显著的结果。

Hundhausen等的论文中评论过AV系统(2002),这些系统根据其教学效果都存在不同的结果。XTango的实验中,仅仅使用教科书与同时使用教科书和可视化工具这两者之间的数据在考试结果上没有差异。另一个关于用XTango和Polka(Byrne,Catrambone,Stasko,1999)作为授课工具的实验中得到的结果是:相比仅仅使用教科书的学生,看过演示动画的学生能够更好地判断出算法的下一步。而在编程方面,使用BALSA II (brown,1988)实际操作并观看过算法演示动画的学生比没有的学生表现得更好。该实验的作者指出当前研究的AV系统都是Java版本前的变形,而且学生只能在计算机实验室中使用电脑。

也许Hundhausen等(2002)的元分析中最重要的成果在于(a)学生如何使用AV对教学效果有影响,而不是学生看到了什么样的AV;以及(b)AV技术能够吸引学生主动交互时最有效果。这种关键在于交互的观点很大程度上影响了AV之后的发展。
\end{sectext}
