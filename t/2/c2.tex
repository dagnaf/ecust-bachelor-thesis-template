% !TEX root = ../../t.tex
\chapter{互联网时代下的AV}
\begin{sectext}
随着万维网和我们熟知的互联网的出现,20世纪90年代中期发生了许多改变。越来越多的教师和学生都在使用AV系统,大多是因为Java的广泛使用,还有一小部分是由于Javascript和Flash。也就大约在这个时期,越来越多的学生在校园里拥有自己的计算机。而从20世纪90年代末到21世纪初,一种新的基于JAVA的AV开发系统逐渐形成,而以前基于X Windows的AV系统在教育领域内几乎完全失去了影响力。

基于JAVA的AV系统包括JSamba(Stasko 1998)、JAWAA(Pierson,Rodger,1998)、JHAVE`(Naps,Eagan,Norton,2000)、ANIMAL(Rossling,Schuler,Freisleben,2000)和TRAKLA2(Korhonen等,2003)。这些系统的共同特点在于,作为开发AV的完整工具,能够为AV开发者提供一个编程框架。Java版本前的系统难以深入开发,虽说投入大量精力后还是能够进行开发的。相反,一些基于Java的开发系统(例如JAWAA)则注重可视化开发的便捷性,但不太重视交互性。

与Java广泛使用起到同样关键性的变化还在于出现了许多不依赖于开发系统的AV。这是因为Java不受制于操作系统或者其他类似的限制,同时互联网也利于发布软件。这就改变了开发者的自身角色。虽然JHAVE和ANIMAL按照传统开发系统,系统还是非常庞大。而其他许多项目则将AV打包为独立软件,不能用于其他AV开发中。大型非打包的系统包括数据结构浏览器(Data Structure Navigator,DSN)(Dittrich,an den Bercken,Schafer,Klein,2001),交互式数据结构可视化(Interactive Data Structure Visualization,IDSV)(Jarc,1999),算法动画(Stern,2001),可视化的数据结构(Galles,2006)和TRAKLA2(Korhonen等,2003)。

Java出现的另一个影响在于教师学生开始进行一次性的AV开发。编写AV对于学生而言是学好JAVA的一种途径,这在90年代末非常吸引人。遗憾的是大多数这些学生编写的AV难以发挥教学作用(Shaffer等,2010)。不过还是有一部分AV是通过几年持续努力而编写出的Java小应用程序,并且在某些课程上展示而受到广泛好评。例如二叉树之类(Binary Treesome)(Gustafson,Kjensli,Vold,2011),JFLAP(Rodger,2008),马里兰大学的空间索引示例(Spatial Index Demos)(Brabec,Samet,2003)和弗吉尼亚理工大学算法可视化研究小组(Virginia Tech Algorithm Visualization Research Group)(2011)。
\end{sectext}
