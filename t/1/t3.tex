% !TEX root = ../../t.tex
\begin{center}
\setstretch{1.35}
\vspace{-20pt}
\begin{longtable}{p{7cm} p{7cm}}
\caption{算法可视化的效果}
\label{tab:3} \\

\hline \multicolumn{1}{l}{积极方面} & \multicolumn{1}{l}{消极方面} \\ \hline
\endfirsthead

\multicolumn{2}{r}%
{接表 \ref{tab:3}} \\
\hline \multicolumn{1}{l}{积极方面} & \multicolumn{1}{l}{消极方面} \\ \hline
\endhead

\hline \multicolumn{2}{r}{待续}
\endfoot

\hline
\endlastfoot

算法可视化能直接给出正确答案(在修复程序错误后)。 & 算法可视化的程序错误在早期测试使用中会影响理解。\\
简介的用户界面,易于使用。 & 一些功能对用户还不够友好(还缺少合适的教程文档)。\\
用户能使用自己的例子:输入具体数据、绘制图和多边形。 & 在CS3233教学中需要的算法可视化不仅仅是上述的九种。\\
有助于更好地理解算法。 & 有了算法可视化可能导致学生不主动练习。\\
比徒手模拟更快。 & 可视化的每个步骤最好配上算法的(伪)代码。\\
\end{longtable}
\end{center}

\vspace{-42pt}
